\documentclass[12pt, a4paper]
{article}

\usepackage{svg}
\setsvg{inkscape=inkscape -z -D,svgpath=images/}
\usepackage{float}
\usepackage{amsmath}
\usepackage{todonotes}

\providecommand{\todos}[1]{\todo[inline]{#1}}
\providecommand{\ang}[1]{\theta_{\text{#1}}}
\providecommand{\sub}[1]{_{\text{#1}}}
\providecommand{\subi}[1]{_{\text{#1},i}}
\providecommand{\twonorm}[1]{\left|\left|#1\right|\right|}
\providecommand{\foralli}{\quad \forall\, i}
\providecommand{\lr}[1]{\left( {#1} \right)}

\providecommand{\tr}{\tilde{r}}

\title{Reference frame conversions}
\author{Laurens Valk}
\begin{document}
\maketitle

\tableofcontents

\section{From Pixels to Meters} 
\subsection{Transformation Matrices}
The symbol $\tr$ expresses vectors in units of pixels. Any other vector $r$ has units of meters.
\begin{figure}[H]
\centering
\includesvg[width=1\textwidth]{scaling}
\caption{Four coordinates: a) in the camera picture, b) with flipped y axis, c) with the origin at the center of the picture, d) scaled to physical distances.
\label{fig:robotframe}}
\end{figure}

\begin{equation}
\begin{bmatrix}
\tr^f\\
1
\end{bmatrix} = \begin{bmatrix}
1 & 0 & 0\\
0 & -1 & 0\\
0& 0 & 1
\end{bmatrix}\begin{bmatrix}
\tr^p\\1
\end{bmatrix}=H^f_p \begin{bmatrix}
\tr^p\\1
\end{bmatrix}
\end{equation}
\begin{equation}
\begin{bmatrix}
\tr^c\\
1
\end{bmatrix} = \begin{bmatrix}
1 & 0 & -\frac{1}{2}\tilde{w}_x\\
0 & 1 & \frac{1}{2}\tilde{w}_y\\
0& 0 & 1
\end{bmatrix}\begin{bmatrix}
\tr^f\\1
\end{bmatrix}=H^c_f \begin{bmatrix}
\tr^f\\1
\end{bmatrix}
\end{equation}
\begin{equation}
\begin{bmatrix}
r^s\\
1
\end{bmatrix} = \begin{bmatrix}
\varepsilon & 0 & 0\\
0 & \varepsilon & 0\\
0& 0 & 1
\end{bmatrix}\begin{bmatrix}
\tr^c\\1
\end{bmatrix}=H^s_c \begin{bmatrix}
\tr^c\\1
\end{bmatrix}
\end{equation}

\begin{equation}
\begin{bmatrix}
r^s\\
1
\end{bmatrix} = H^s_c H^c_f H^f_p \begin{bmatrix}
\tr^p\\1
\end{bmatrix}
\end{equation}
%
The overall transformation from pixels to the real world representation is
%
\begin{equation}
\begin{bmatrix}
r^s\\
1
\end{bmatrix} = H^s_p \begin{bmatrix}
\tr^p\\1
\end{bmatrix}
\end{equation}
%
where
%
\begin{equation}
H^s_p =H^s_c H^c_f H^f_p = \begin{bmatrix}
 \varepsilon&    0& -\frac{1}{2}\varepsilon \tilde{w}_x\\
   0& -\varepsilon&  \frac{1}{2}\varepsilon \tilde{w}_y\\
0&    0&           1
\end{bmatrix}
\end{equation}



\subsection{Scaling Factor}
%
The distance $d\sub{top/center}$ in Figure \ref{fig:robotframe} is known, constant, and equivalent for each agent:
%
\begin{equation}
d\sub{top/center} \equiv \twonorm{r\subi{top}-r\subi{center}} \foralli
\end{equation}
%
Per agent, we can estimate the scaling factor $\epsilon_i$ as:
%
\begin{equation}
\varepsilon_i = \dfrac{d\sub{top/center}}{\twonorm{\tr\subi{top}-\tr\subi{center}}} \quad \text{m/pixel}
\end{equation}
%
A more accurate estimate for the overall scaling factor is:
%
\begin{equation}
\varepsilon = \dfrac{1}{N}\sum^N_{i=1} \varepsilon_i
\end{equation}
%
However, because $\varepsilon$ is constant it may be more desirable to calculate it once and use it as a constant parameter. If so, we may divide the arena width by the picture with in pixels:
%
\begin{equation}
\varepsilon = \dfrac{d\sub{field}}{\tilde{w}_x} \quad \text{m/pixel}
\end{equation}

\section{Local Robot Frames}
\subsection{Label Orientation}

\begin{figure}[H]
\centering
\includesvg[width=1\textwidth]{robotframe}
\caption{Local vehicle frame definitions
\label{fig:robotframe}}
\end{figure}
\todos{fix $\theta\sub{label}$ to fixed value to avoid computing sine/cos explicitly to get rotation matrix.}
\todos{Define homogeneous transformation matrices between all steps to account for both displacement and rotation}

\section{Relative Agent Positions and Orientations}




\end{document}